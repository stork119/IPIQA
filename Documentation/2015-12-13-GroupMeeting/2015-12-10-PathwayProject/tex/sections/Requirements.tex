\section{Wymagania i problemy}

Problemy i ograniczenia związane z projektem : 
\begin{itemize}
\item Czas obrazowania przez mikroskop konfokalny w przypadku 96 dołkowego eksperymentu ~12h
\item Czas kwantyfikacji danych ~12h
\item Różne konfiguracje eksperymentów :
\begin{itemize}
\item Liczba odczytywanych kanałów.
\item Różna liczba oraz pozycja analizowanych dołków
\item Konieczność łączenia eksperymentów
\end{itemize}
\item Błędy w obrazowaniu - powtórzenia dla pojedynczych dołków.
\item Konieczność stosowania różnych metod do kwantyfikacji danych (różne pipeliny w CellProfilerze)
\item Niektóre algorytmy wymagają informacji o całym zbiorze danych.
\item Konieczność przesyłu plików
\item Ograniczenia dyskowe - należy umiejętnie zarządzać plikami
\item Wielkość przetrzymywanych danych (obecnie dane są przetrzymywane w plikach csv)
\end{itemize}

W wyniku tych problemów należy zaimplementować oprogramowanie, które spełnia następujące wymagania:
\begin{itemize}
\item Proste i intuicyjny system do zapisywania schematu eksperymentu.
\item Prosta modyfikowalność Workflowu - bez konieczności zmian w implementacji programu.
\item Możliwość pomijania niektórych kroków (kwantyfikacji, korekty obrazu, itd.)
\item Zrównoleglanie operacji 
\begin{itemize}
\item obrazowanie - kwantyfikacja
\item kwantyfikacja - kwantyfikacja
\end{itemize}
\item Oddzielanie kolejnych etapów procedowania na zasadzie input-output. 
\item 
\item
\end{itemize}


