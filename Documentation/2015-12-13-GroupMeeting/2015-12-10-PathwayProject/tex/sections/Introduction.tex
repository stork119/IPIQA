\section{Wstęp}


W ramach realizowanych projektów wykonywane są eksperymenty z wykorzystaniem metod immunocytochemicznych.
Wynikiem takich eksperymentów są zdjęcia komórek, których odpowiednie białka są znakowane fluorescencyjnie.
Zdjęcia te są  otrzymane  za pomocą obrazowania wykonanego przy użyciu mikroskopu konfokalnego wysokiej przepustowości BD Pathway 435 Bioimager. 
Umożliwia on na jednoczesną, automatyczną analizę dużej liczby eksperymentów, które znajdują się na płytce eksperymentalnej. 
Płytka pokryta jest dołkami, w których przetrzymywane są komórki poddane eksperymentom. Każdy dołek odpowiada jednemu eksperymentowi.
Aby uściślić nazewnictwo w dalszej części przyjmujemy, że
 \begin{itemize}
\item eksperyment - oznacza całą serię eksperymentów znajdujących się na jednej płytce,
\item dołek - pojedynczy eksperyment znajdujący się w jednym dołku na płytce.
\end{itemize}

Otrzymane obrazy są następnie poddane analizie w celu uzyskania kwantyfikacji obrazu. Analiza stosuje zaawansowane algorytmy dzięki, którym badane są komórki pod względem stężenia danego białka (odpowiednio w cytoplazmie) oraz opis morfologiczny, jak i populacji. 
Projektowane oprogramowanie ma przy użyciu innych narzędzi programistycznych automatyzować cały proces kwantyfikacji jak i analizy wyników.

Obecne oprzyrządowanie i związane z tym ograniczenia umożliwiają przygotowanie eksperymentu z 96 dołkami, ale możliwości mikroskopu są znacznie większe.
Plany związane z realizacją grantu ... przewidują zakup narzędzi, które będą umożliwiały wykonanie eksperymentów złożonych z 192 dołków, stąd konieczność automatyzacji całej procedury.

