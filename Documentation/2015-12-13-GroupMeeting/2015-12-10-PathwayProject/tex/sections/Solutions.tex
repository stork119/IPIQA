\section{Rozwiązania}

\begin{enumerate}
\item Rozwój obecnego oprogramowania w języku R. Oprogramowanie stosowane dzisiaj, tworzone i rozwijane na doraźne potrzeby automatyzuje całą opisaną procedurę, jednakże nie zapewnia wszystkich opisanych wymagań. W szczególności nie zrównolegla operacji oraz nie nie zarządza przetrzymywanymi plikami. Zmiany w projektowanych eksperymentach i workflowle wymagają zmian w implementacji.
Dostosowanie tego oprogramowania do stawianych wymogów byłoby równie czasochłonne, jak stworzenie implementacji od nowa. 
Ponadto język oprogramowania R użyty do stworzenia projektu nie jest właściwy do spełnienia wszystkich wymagań.
Nie mniej znaczna część programu - zwłaszcza związana z analizą danych oraz tworzeniem wykresów może w łatwy sposób zostać wykorzystana w projekcie.


\item Pakiet do Cell Profillera. Stworzenie nowego pakietu (bądź pakietów) do wolnego oprogramowania Cell Profiller byłoby najkorzystniejszym rozwiązaniem ze względu na możliwość stworzenia publikacji i udostępnienia oprogramowania na zewnątrz. Jednakże dostosowanie rozwiązania do wymogów stawianych przez projektantów Cell Profillera może być bardzo ograniczające i czasochłonne w implementacji.

\item Stworzenie nowej implementacji w Pythonie. 
Rozwiązanie pośrednie pomiędzy rozwojem własnego oprogramowania, a tworzeniem pakietu do Cell Profillera zakłada implementację nowego programu w języku Python w taki sposób, aby istniała możliwość dostosowania poszczególnych modułów funkcjonalnych do wymagań Cell Profillera. 
\end{enumerate}