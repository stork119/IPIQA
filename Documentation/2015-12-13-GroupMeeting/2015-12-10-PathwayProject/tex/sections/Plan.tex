\section{Plan pracy}

Poniżej przedstawiony jest ogólny plan prac. Prace będą realizowane przez dwie osoby : Agnieszkę Gromadkę (AG) oraz Karola Nienałtowskiego (KN)

\begin{enumerate} 
\item Grudzień
\begin{itemize}
\item KN Projekt programu
\item KN Moduł Cell Profiler
\item AG Moduł File Management
\item KN-AG Konfiguracja sieciowa komputera obsługującego Pathwaya
\item KN-AG Konfiguracja systemu zarządzania projektem 
\item AG Stworzenie struktury projektu
\item AG Konfiguracja oprogramowania
\item KN-AG
\end{itemize}
\item Styczeń
\begin{itemize}
\item KN-AG Projekt struktury workflowu
\item AG Moduł XML parser
\item AG Moduł TASK
\item KN Projekt struktury MAP PLATE'a
\item AG Moduł do parsowania MAP PLATE'a
\item KN Moduł Analiza Danych
\item KN Funkcja do integracji danych z Cell Profilera
\end{itemize}
\item Luty
\begin{itemize}
\item AG Moduł R Connection
\item AG Moduł Python Connection
\item KN Moduł  ImageJ
\item AG Analiza danych - przygotowanie statystyk
\end{itemize}
\item Marzec
\begin{itemize}
\item AG Moduł Correction
\item KN-AG Wdrożenie projektu. Dokumentacja. Wersja 0.1
\end{itemize}
\item Kwiecień
\begin{itemize}
\item AG Moduł do tworzenia rysunków
\item AG Analiza danych - korekta morfologiczna
\end{itemize}
\item Maj
\begin{itemize}
\item KN-AG Zarządzanie plikami wprowadzenie poprawek do projektu
\item AG Wdrożenie projektu wersja 1.0
\end{itemize}
\end{enumerate}

\begin{lstlisting}
function [Cmatrix, 	...
             	Qmatrix, 	...
              	Wmatrix,	 ...
               	IWmatrix,	 ...
               	Ematrix,	 ...
               	timer,	...
               	n, 	...
               	Bmatrix] = ArimotoBlahutAlgorithm( P, Q, N, eTollerance )
%% Arimoto Blahut Algorithm
%% INPUT
% P = P(Y|X) -- model rows output; columns input
% Q = P(X)   -- a priori
% N  -- limit of evaluation -- default 1000
% eTollerance -- desired accuracy of the algorithm -- default 0.001

%% OUTPUT 
% Cell arays  1 x N
% Cmatrix -- estimation of capacity channel I(Q*,W*)
% Qmatrix -- Q* estimated distribution of parameters
% Wmatrix -- P(X|Y) posteriori distrbution
% Imatrix  -- I(Q*)
% Ematrix -- terminate condtition ln(Q^(n+1)/Q^(n))
% Bmatrix -- number of states
%% Private functions

% Creates Wmatrix 
Wfun = @(n, Pmatrix, Qprob) ...
    ((Pmatrix*diag(Qprob))./ ...
    repmat(Pmatrix*(Qprob)', 1, size(Pmatrix, 2)))';

% Creates Amatrix
alphaFun = @(Pmatrix, Wmatrix) diag(exp(logzerom(Wmatrix, Pmatrix, false)))';


% Mutual Information
Ij = @(Pmatrix, Qprob) sum(logzerov(Pmatrix.*Qprob, Pmatrix./(Pmatrix*Qprob')));
I = @(Pmatrix, Qprob) ...
      summarize(@(j) Ij(Pmatrix(j,:), Qprob), {1:size(Pmatrix,1)});
%% Initialization
if nargin < 4
  if nargin < 3
      eTollerance = 0.001;
  end
  N  = 1000;
  
end

Wmatrix = cell(1, N);
Amatrix = cell(1, N);
Cmatrix = cell(1, N);
Bmatrix = cell(1, N);
Ematrix = cell(1, N);

Qmatrix = cell(1, N);
Qmatrix{1} = Q;

Pmatrix = P;

Imatrix = cell(1, N);

IWmatrix = cell(1, N);

Cmatrix{1} = I(Pmatrix, Qmatrix{1});
Bmatrix{1} = exp(Cmatrix{1});
%% Implementation

timer = cell(1, N);

terminateAlgorithm = false;
n = 0;
while ~terminateAlgorithm 
  n = n + 1;
  timerTmp = zeros(1,5);
  disp(['Loop ', num2str(n)]);
  tic;
  Wmatrix{n} = Wfun(n - 1, Pmatrix, Qmatrix{n}); 
  timerTmp(1) = toc;
  tic;
  Amatrix{n} = alphaFun(Pmatrix, Wmatrix{n});
  timerTmp(2) = toc;
  tic;
  Qmatrix{n + 1} = Amatrix{n} ./ sum(Amatrix{n});
  timerTmp(3) = toc;

  tic;
  Cmatrix{n + 1} = log(sum(Amatrix{n}));
  Bmatrix{n + 1} = exp(Cmatrix{n + 1});
  timerTmp(4) = toc;
  disp(['I(Q, W^n) ', num2str(Cmatrix{n+1})]);
  
  tic;
  Ematrix{n + 1} = -Cmatrix{n+1} + max(log(Amatrix{n}) - log(Qmatrix{n}));
  timerTmp(5) = toc;
  disp(['End condition ', num2str(Ematrix{n+1})]);
  timer{n + 1} = timerTmp;
  terminateAlgorithm = Ematrix{n + 1} < eTollerance || n + 1 == N;
end

end
\end{lstlisting}
